\documentclass[german,a4paper,12pt]{article}
\renewcommand{\familydefault}{\sfdefault}
\renewcommand{\footnotesize}{\normalsize}
\newcommand{\doctitle}{Dokumententitel}
\newcommand{\doctype}{Dokumententyp}
\newcommand{\docauthor}{Dokumentenautor}
\newcommand{\docauthorid}{Matrikelnummer}
\usepackage{nameref}
\makeatletter
\newcommand*{\sectionname}{\@currentname}
\makeatother
\usepackage[utf8]{inputenc}
\usepackage[T1]{fontenc}
\usepackage[ngerman]{babel}
\usepackage{acronym,array,blindtext,caption,csquotes,enumitem,float,graphicx,parskip,pdfpages,tikz}
\usepackage[
	backend=biber,
	bibwarn=true,
	bibencoding=utf8,
	sorting=none,
	citestyle=verbose,
	style=numeric,
]{biblatex}
\setcounter{biburllcpenalty}{7000}
\setcounter{biburlucpenalty}{8000}
\bibliography{literature.bib}
\usepackage{fancyhdr}
\fancyhf{}
\pagestyle{fancy}
\lfoot{\doctype~-~\docauthorid}
\renewcommand{\sectionmark}[1]{\markboth{#1}{}}
\lhead{\raisebox{5pt}{\nouppercase{\leftmark}}}
\rfoot{\thepage}
\rhead{\includegraphics[width=20pt]{img/util/logo-small.png}}
\setlength{\headheight}{25pt}
\setlength{\headwidth}{427pt}
\setlength{\parindent}{0em}
\setlength{\parskip}{15pt}
% \setlength{\footnotesep}{25pt}
\usepackage[bottom, perpage]{footmisc}
\usepackage[onehalfspacing]{setspace}\onehalfspacing
\usepackage{scrextend}\deffootnote[0em]{0em}{0em}{\textsuperscript{\thefootnotemark}\,}
\usepackage{tablefootnote}
\newcolumntype{P}[1]{>{\centering\arraybackslash}p{#1}}
\usepackage[margin=3cm]{geometry}
\usepackage[
	colorlinks,
	pdfpagelabels,
	pdfstartview=FitH,
	linkcolor=black,
	urlcolor=black,
	plainpages=false,
	hypertexnames=false,
	citecolor=black,
	linktocpage=false, %Click on numbers in toc, not on actual text to link
]{hyperref}
\usepackage{listings}
\lstset{
	backgroundcolor=\color{gray},
	rulecolor=\color{black},
	frame=shadowbox,
	aboveskip=3mm,
	belowskip=3mm,
	showstringspaces=false,
	columns=flexible,
	basicstyle={\small\ttfamily},
	numbers=none,
	numberstyle=\tiny,
	keywordstyle=\color{blue},
	commentstyle=\color{darkgreen},
	stringstyle=\color{orange},
	morekeywords=[1]{@},
	keywordstyle=[1]\color{red},
	keywordstyle={[2]\color{blue}},
	keywordstyle=\bfseries\color{purple},
	breaklines=true,
%	breakatwhitespace=true
	tabsize=2,
	numbers=left,
	firstnumber=1,
	stepnumber=1,
}
\usepackage[nottoc,notlof,notlot,numbib]{tocbibind}
\usepackage{xcolor}
\definecolor{darkgreen}{rgb}{0.0,0.5,0.0}
\definecolor{orange}{rgb}{1.0,0.4,0.0}
\definecolor{purple}{rgb}{0.6,0.2,0.9}
\definecolor{darkblue}{rgb}{0.0,0.0,0.5}
\definecolor{gray}{rgb}{0.96,0.96,0.96}
% \chead{\raisebox{5pt}{\large{ENTWURF}}}
\begin{document}
\title{
	\vspace{50pt}
	\texttt{\doctitle}\\
	\vspace{40pt}
	\texttt{\doctype}\\
	\vspace{20pt}
	\small{\texttt{des Studienganges xy}}\\
	\small{\texttt{an der DHBW Stuttgart}}\\
}
\author{
	\small{\texttt{von}}\\
	\texttt{\docauthor}
	\\ \\ \\ \\
	\small{\texttt{\today}}
}
\date{
	\vspace*{\fill}
	\begin{table}[H]\begin{tabular}{@{}p{0.4855\linewidth}p{0.4855\linewidth}@{}}
	\texttt{Abgabedatum:}						& \texttt{}\\
	\texttt{Bearbeitungszeitraum:} 				& \texttt{}\\
	\texttt{Matrikelnummer, Kurs:}				& \texttt{}\\
	\texttt{Dualer Partner:}					& \texttt{}\\
	\texttt{Betreuer des dualen Partners:}		& \texttt{}\\
	\texttt{Betreuer der Dualen Hochschule:}	& \texttt{}
	\end{tabular}\end{table}
}
\maketitle \thispagestyle{empty}
\begin{tikzpicture}[remember picture,overlay]\node[yshift=-25mm,anchor=north] at (current page.north){\includegraphics[width=100mm]{img/util/logo.png}};\end{tikzpicture}
\newpage
\pagenumbering{Roman}\setcounter{page}{1}\vspace*{\fill}
\section*{\centering{Erklärung}}
Ich versichere hiermit, dass ich meine \doctype mit dem Thema \textit{\glqq\doctitle\grqq} selbstständig verfasst und keine anderen als die angegebenen Quellen und Hilfsmittel benutzt habe. Aus den benutzten Quellen direkt oder indirekt übernommene Inhalte habe ich als solche kenntlich gemacht. \par
Diese Arbeit wurde bisher in gleicher oder ähnlicher Form oder auszugsweise noch keiner anderen Prüfungsbehörde vorgelegt und auch nicht veröffentlicht. \par
Falls beide Fassungen gefordert sind versichere ich zudem, dass die eingereichte elektronische Fassung mit der gedruckten Fassung übereinstimmt. \par
\vspace{70pt}
\begin{table}[H]\begin{tabular}{@{}p{0.4\linewidth}P{0.545\linewidth}@{}}
Ort, den \today     	& \hrulefill\\
						& \vspace{1pt}\docauthor\\
\end{tabular}\end{table}
\vspace*{\fill+50pt}
\newpage
\vspace*{\fill}
\section*{\centering{Zusammenfassung}}
\blindtext
\vspace*{\fill}
\newpage
% \vspace*{\fill}
% \section*{\centering{Abstract}}
% \vspace*{\fill}\newpage
\tableofcontents
\newpage
\listoffigures
\newpage
% \listoftables\newpage
% \lstlistoflistings\newpage
% \section*{Abkürzungsverzeichnis}
% \begin{acronym}[onlyused]
% \end{acronym}
\newpage
\pagenumbering{arabic}\setcounter{page}{1}
\section{Blindtext}
\blindtext \par
\blindtext \par
\blindtext \cite{TANENBAUM2016}
\newpage
\section{Elemente}
\subsection{Bilder}
\vspace{10pt}
\begin{figure}[H]
    \centering
    \includegraphics[width=0.8\textwidth]{img/colors.png}
    \caption{Farben}
    \label{fig-adwareinstaller}
\end{figure}
\vspace{10pt}
\subsection{Tabellen}
\begin{table}[H]\begin{tabular}{@{}|p{0.2\linewidth}|p{0.745\linewidth}|@{}}\hline
	\textbf{Spalte 1} & \textbf{Spalte 2} \\\hline
	Inhalt 1 & \blindtext \\\hline
\end{tabular}\end{table}\vspace{10pt}
\subsection{Code}
\vspace{10pt}\begin{lstlisting}[language={C++}, caption={Test mit simulierten Werten (C\raisebox{0.25ex}{++})}, label={source-statistic}]
	if (globalTestFlag == 1)
	{ // Test soll durchgefuehrt werden
			std::cout << "Test has started." << endl;
			std::ifstream file(testFilePath);
			if(!file.is_open())
			{ // Datei mit Werten konnte nicht gefunden werden
					std::cerr << "File not found" << std::endl;
					return;
				}
			string tp;
			sal_Int32 i = 0;
			while(getline(file, tp))
			{ // Read file line by line and store values
					ZResourceRecord rec;
					rec.value = stod(tp);
					rec.timestamp = time(0)-(10*i);
					dataHashMap[ZResource::TEST].push_front(rec);
					i++;
			}			
		}
\end{lstlisting}\vspace{10pt}
\newpage
\printbibliography[title={Literatur}]
\end{document}